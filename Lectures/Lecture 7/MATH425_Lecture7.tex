\documentclass[12pt,a4paper]{article}
\usepackage[utf8]{inputenc}
\usepackage{amsmath}

\usepackage{amsfonts}
\usepackage{amssymb}
\usepackage{tikz}
\usepackage{amsmath}
\usepackage{amssymb}
\usepackage{pgfplots}
\usepackage{nccmath}
\usepackage{mathtools}
\usepackage{pgfplots}
\usepackage{mathtools,amssymb}
\usepackage{tikz}
\usepackage{xcolor}
\pgfplotsset{compat = newest}
\author{Chris Camano: ccamano@sfsu.edu}
\title{MATH 425 Lecture 7 }
\date{2/17/2022}
% Margins
\topmargin=-0.45in
\evensidemargin=0in
\oddsidemargin=0in
\textwidth=6.5in
\textheight=9.0in
\headsep=0.25in
\newcommand{\q}{\quad}
\renewcommand{\labelenumi}{\alph{enumi})}
\newcommand{\R}{\mathbb{R}}
\newcommand{\rtwo}{$\mathbb{R}^2$}
\newcommand{\C}{$\mathbb{C}$}

\begin{document}
\maketitle

\section*{Refresher on linear independence}
Setting a matrix equal to x,y,z yields the plane or three dimensional space spanned by the vectors of that matrix.
\section*{Linear independence and dependence}\\
A set of vectors is regarded to be linearly depedent if there exists a solution to the equation \textbf{Ax=0} that is not the trivial solution.

\[
  \text{ A is linearly dependent }\iff \exists x | Ax=0, x\neq 0
\]
\\
This is to say that there exists some weights for the columns of A such that the sum of the weighted vectors is equal to zero.


\end{document}
