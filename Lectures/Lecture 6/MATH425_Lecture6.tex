\documentclass[12pt,a4paper]{article}
\usepackage[utf8]{inputenc}
\usepackage{amsmath}

\usepackage{amsfonts}
\usepackage{amssymb}
\usepackage{tikz}
\usepackage{amsmath}
\usepackage{amssymb}
\usepackage{pgfplots}
\usepackage{nccmath}
\usepackage{mathtools}
\usepackage{pgfplots}
\usepackage{mathtools,amssymb}
\usepackage{tikz}
\usepackage{xcolor}
\pgfplotsset{compat = newest}
\author{Chris Camano: ccamano@sfsu.edu}
\title{MATH425 Lecture 6 }
\date{2/15/2022}
% Margins
\topmargin=-0.45in
\evensidemargin=0in
\oddsidemargin=0in
\textwidth=6.5in
\textheight=9.0in
\headsep=0.25in
\newcommand{\q}{\quad}
\renewcommand{\labelenumi}{\alph{enumi})}
\newcommand{\R}{\mathbb{R}}
\newcommand{\Rtwo}{\mathbb{R}}
\newcommand{\C}{\mathbb{C}}

\begin{document}
\maketitle

\section{Discussion of linear independence and Span}
Given a set of vectors of the form
\[
  \{v_1,v_2,v_3\}
\]
Span$  \{v_1,v_2,v_3\}$ is the region spanned by all linear combination of these vectors. \\
Discussion of linear systems: Given two equations of the from
\[
  2x-y=1, x+y=5
\]
We can augment these equations to a matrix of the form
\[
  \begin{bmatrix}
    2&-1&1\\1&1&5
  \end{bmatrix}
\]
\subsection{Row space view}
understanding this matrix throught the perspective of row space we can understand the matrix as a series of rows$\{ v_{11},v_{21},..,v_{n1}\}$ dotted with \begin{bmatrix}
  x_1\\x_2\\...\\x_n
\end{bmatrix}
such that
\[
  \{ v_{1,1},v_{2,1},..,v_{n-1,1}\} \cdot \begin{bmatrix}
    x_1\\x_2\\...\\x_n
  \end{bmatrix}= y_{1,n}
\]
In the context of this exmple we have
\[\begin{bmatrix}
    2&-1
  \end{bmatrix} \cdot \begin{bmatrix}
    x_1\\x_2
  \end{bmatrix}= 1
\]
Through the row space view we can think of each row as a linear equation with the intersection of the rows representing the solution to the system of linear equations. //
\subsection{Column Space view}
Express the columns of the matrix as a linear combination such that the final vector is the product of the sum of these weighted columns:
\[
  x\begin{bmatrix}
    2\\1
\end{bmatrix}+  y\begin{bmatrix}
    -1\\1
\end{bmatrix}=  \begin{bmatrix}
    1\\5
\end{bmatrix}
\]
Essentially this is the process of thinking of stretching out vectors such that the sum of their now stretched version equates to the desired vector.//
\subsection{Span}
Span is the vector space formed by all linear combinations of a set of vectors.
\[
  \text{Let } \{v_1,v_2,...,v_n\} \in \mathbb{F}^n.
\]
\[
  \text{Span } \{v_1,v_2,...,v_n\}
\]
is the set of all vectors of the form:
\[
  c_1v_1+c_2v_2+...+c_nv_n
\]
\[
  c_1,c_2,...c_n \in \mathbb{F}^n
\]
\[
  \text{Span }\{v_1,v_2,...,v_n\} \subset \mathbb{F}^n
\]
\end{document}
