\documentclass[12pt]{article}
\usepackage[pdftex]{graphicx}
\usepackage{amsmath}
\usepackage{amssymb}
\pagestyle{empty}
\author{Chris Camano: ccamano@sfsu.edu}
\title{MATH 425  Lecture 11 }
\date{3/3/2022}

\topmargin -0.6in
\headsep 0.40in
\oddsidemargin 0.0in
\textheight 9.0in
\textwidth 6.5in

\newcommand{\econst}{\mathrm{e}}
\newcommand{\diff}{\mathrm{d}}
\newcommand{\dwrt}[1]{\frac{\diff}{\diff #1}}
%%%%%%Macros for 425%%%%%%%%%%%%%%%%%%%
\newcommand{\q}{\quad}
\newcommand{\tab}{\\\\}
\renewcommand{\labelenumi}{\alph{enumi})}
\newcommand{\sect}[1]{\section*{#1}}

%%%%%%Vector Spaces%%%%%%%%%%%%%%%%%%%
\newcommand{\R}{\mathbb{R}}
\newcommand{\C}{\mathbb{C}}
\newcommand{\F}{\mathbb{F}}
\newcommand{\rtwo}{\mathbb{R}^2}
\newcommand{\mxn}{{mxn}}

%%%%%%Sets and common phrases%%%%%%%%%
\newcommand{\Axb}{\textbf{Ax=b} }
\newcommand{\Axz}{\textbf{Ax=0} }
\newcommand{\dim}{\text{dim}}
\newcommand{\lc}{linear combination }
\newcommand{\let}{\text{Let }}
\newcommand{\tf}{\therefore}
%%%%%%%%%%%%%%%%%%%%%%%%%%%%%%%%%%%%%%%
\everymath={\displaystyle}

\begin{document}
\maketitle
\sect{Review of homework}
You can only talk about eigenvalues in the context of square matrices or linear operators.
\textbf{Try researching linear transformations that operate as dimensionality reduction}This is in some ways a projection to a lower dimensional subspace.

\sect{Homegenous Coordinates}
One methodology for evaluating whether or not a transformation is a linear transforamtion is examining the effect the transformation has on the zero vector. A transforamtion is not linear if the zero vector is mapped to some other location. \\
\textbf{Homogenus Coordinates}: A method for preforming translation via matrix multiplication. This can be used to translate a two dimensional place through a three dimensional space. :
\[
  \text{ Given }x,y,\in \R^2\quad  \forall a,z \in \R, z\neq 0 \quad (xz,yz,z)
\]
are the Homogenus Coordinates of xy.
\newcommand{\hg}{Homogenus Coordinates }
\\Suppose $(x,y)=(2,3)$ the \hg of (2,3) is (2,3,1)
\\ This concept generalizes to higher dimensions as well:
\[
    \text{ Given }x,y,z\in \R^3\quad  \forall h\in \R, h\neq 0 \quad (xh,yh,zh,h)
\]
are the Homogenus Coordinates of xyz.\\
\\
Translation can be understood as a form of matrix multiplication: \\
Recall for a translation by some vector $v=\begin{bmatrix}
  h\\k
\end{bmatrix}$
\[
  \begin{bmatrix}
    x\\y
  \end{bmatrix}
  \mapsto
  \begin{bmatrix}
    x\\y
  \end{bmatrix}+
  \begin{bmatrix}
    h\\k
  \end{bmatrix}=
  \begin{bmatrix}
    x+h\\y+k
  \end{bmatrix}
\]
which is the transformation that desicribes the translation of x by vector v\\
The matrix multiplication is:
\[
\begin{bmatrix}
1&0&h\\
0&1&k\\
0&0&1\\
\end{bmatrix}
\begin{bmatrix}
  x\\y\\1
\end{bmatrix}=
\begin{bmatrix}
  x+h\\y+k\\1
\end{bmatrix}
\]
Let H=
\[
\begin{bmatrix}
\mathbb{I} &v\\
\vec{0}^T & 1
\end{bmatrix}
\]
Any linear transformation on $\R^n$ is represented with respect to
\hg by a partitioned matrix of the form:
\[
H=
\renewcommand{\arraystretch}{1.5}
\begin{bmatrix}
A &\vec{0}\\
\vec{0}^T & 1
\end{bmatrix}
\]
Where A is the nxn matrix of the transformationin $\R^n$\\

\textbf{2d scalar matrix in \hg}
\\
let A= $\begin{bmatrix}
  k&0\\0&h
\end{bmatrix}$
\[
\renewcommand{\arraystretch}{1.5}
\begin{bmatrix}
A &\vec{0}\\
\vec{0}^T & 1
\end{bmatrix}
\]
\textbf{2d rotation matrix by theta in \hg}
\\
let A= $\begin{bmatrix}
  \cos(\theta)&-\sin(\theta)\\\sin(\theta)&\cos(\theta)
\end{bmatrix}$
\[
\renewcommand{\arraystretch}{1.5}
\begin{bmatrix}
A &\vec{0}\\
\vec{0}^T & 1
\end{bmatrix}
\]

\textbf{2d rotation matrix by theta in \hg}
\\
\end{document}
