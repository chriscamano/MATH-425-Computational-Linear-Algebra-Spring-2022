\documentclass[12pt,a4paper]{article}
\usepackage[utf8]{inputenc}
\usepackage{amsmath}

\usepackage{amsfonts}
\usepackage{amssymb}
\usepackage{tikz}
\usepackage{amsmath}
\usepackage{amssymb}
\usepackage{pgfplots}
\usepackage{nccmath}
\usepackage{mathtools}
\usepackage{pgfplots}
\usepackage{mathtools,amssymb}
\usepackage{tikz}
\usepackage{xcolor}
\pgfplotsset{compat = newest}
\author{Chris Camano: ccamano@sfsu.edu}
\title{MATH425 Lecture 5 }
\date{2/10/2022}
% Margins
\topmargin=-0.45in
\evensidemargin=0in
\oddsidemargin=0in
\textwidth=6.5in
\textheight=9.0in
\headsep=0.25in
\newcommand{\q}{\quadd}
\renewcommand{\labelenumi}{\alph{enumi})}
\newcommand{\rtwo}{$\mathbb{R}^2$}
\newcommand{\C}{$\mathbb{C}$}

\begin{document}
\maketitle
\section{The inner product/dot product}
Given two vectors:
\[
  \vec{v}=[2,2]
\]
\[
  \vec{u}=[3,0]
\]
The dot product of those two vectors an be expressed as the following generalization:
\begin{align*}
    \text{Given } v=[v_1,....v_n]\\
    \text{and } u= [u_1,...,u_n]\\
    v \cdot u=\sum_{i=1}^{n}v_iu_i
\end{align*}
Two vectors are \textbf{orthogonal}if it is the case that:
\[
  \vec{u} \cdot \vec{v} =0 , \forall \vec{u},\vec{v} \in \mathbb{F}^d
\]

\]
\end{document}
