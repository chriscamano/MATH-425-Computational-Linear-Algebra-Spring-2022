\documentclass[12pt]{article}
\usepackage[pdftex]{graphicx}
\usepackage{amsmath}
\usepackage[final]{pdfpages}


\usepackage{amssymb}
\pagestyle{empty}
\author{Chris Camano: ccamano@sfsu.edu}
\title{MATH 425  Lecture 15 }
\date{3/25/2022}

\topmargin -0.6in
\headsep 0.40in
\oddsidemargin 0.0in
\textheight 9.0in
\textwidth 6.5in

\newcommand{\econst}{\mathrm{e}}
\newcommand{\diff}{\mathrm{d}}
\newcommand{\dwrt}[1]{\frac{\diff}{\diff #1}}
%%%%%%Macros for 425%%%%%%%%%%%%%%%%%%%
\newcommand{\q}{\quad}
\newcommand{\tab}{\\\\}
\renewcommand{\labelenumi}{\alph{enumi})}
\newcommand{\sect}[1]{\section*{#1}}

%%%%%%Vector Spaces%%%%%%%%%%%%%%%%%%%
\newcommand{\R}{\mathbb{R}}
\newcommand{\C}{\mathbb{C}}
\newcommand{\F}{\mathbb{F}}
\newcommand{\Z}{\mathbb{Z}}
\newcommand{\rtwo}{\mathbb{R}^2}
\newcommand{\mxn}{{mxn}}

%%%%%%Sets and common phrases%%%%%%%%%
\newcommand{\Axb}{\textbf{Ax=b} }
\newcommand{\Axz}{\textbf{Ax=0} }
\newcommand{\dim}{\text{dim}}
\newcommand{\lc}{linear combination }
\newcommand{\let}{\text{Let }}
\newcommand{\tf}{\therefore}
%%%%%%%%%%%%%%%%%%%%%%%%%%%%%%%%%%%%%%%
\everymath={\displaystyle}


\begin{document}
\maketitle

\sect{Opening Notes}
Professor Hosten will be visiting next class. \\
\sect{Orthogonality}
\textbf{Unit Vector}\\ A unit vector is a vector of length one. \\
To find a unit vector that is a basis for a subspace you first take a given vector in the subspace and normalize. This returns a unit vector for the subspace. \\\\
\textbf{Orthogonal subspace $W^\perp$}\\ \[
  \forall \text{ Subspaces }W, \exists W^\perp : \forall w\in W, w^\perp \in W^\perp\quad  w \cdot w^\perp=0
\]
This is to say for all subspaces there is an Orthogonal complement in which all vectors from the first subspace dotted with vectors in the orthogonal complement are 0.\\\\
\textbf{Orthogonal complement }\\ The orthogonal complement of  a subspace is called the orthogonal complement. The orthogonal complement is the set of all vectors that are orthogonal to W.\\\\
\[
  \text{RowA}^\perp=\text{NulA}\quad \text{ColA}^\perp=\text{NulA}^T
\]
\textbf{Orthogonal sets}\\
A set of vectors is an orthogonal set if each pair of distinct vectors from the set is orthogonal. \\\\
\textbf{Orthonormal set}\\
An orthonormal set is a set of orthogonal vectors that are normalized. \\\\
An mxn matrix U has orthonormal columns iff :
\[
  U^TU=\mathbb{I}_n
\]
Proof:\begin{proof}
$$
  \text{Let } U = \begin{bmatrix}
    |&|&|\\
    $u_1$&$u_2$&...$u_3$\\
    |&|&|\\
\end{bmatrix}
$$
$u_i\in \F^n$
\[
  U^TU=\begin{bmatrix}
    u_1^Tu_1&...& u_1^Tu_m\\
    .&&.\\
    .&&.\\
    u_m^Tu_1&...& u_n^Tu_n
\end{bmatrix}=\mathbb{I}_n
\]
as $\forall u_i, u_i\in U u_i^Tu_i=1$
\end{proof}\\\\
\textbf{Orthogonal matrix}\\
A square matrix U for which $U^-1=U^T$ is called an orthogonal matrix. \\\\
\textbf{Orthongality and linear independence }:
If S =\{$u_1,....,u_p\}$ is an on orthongal set of non zero vectors in $\F^n$ then S  is linearly independent and is a basis for the subspace spanned by S. \\\\
\textbf{Orthogonal basis}\\ An orthogonal basis for a subspace W of $\R^n$ is  a basis for W that is also an orthogonal set.\\
\textbf{theorem}:\\
Let U be an mxn matrix with Orthonormal columns and let x and y be in $\R^n$ then:
\begin{itemize}
  \item $Ux \cdot Uy=x\cdot y$
  \item $||Ux||$=$||x||$
  \item $Ux \cdot Uy$=0
\end{itemize}\\
\textbf{}
\end{document}
