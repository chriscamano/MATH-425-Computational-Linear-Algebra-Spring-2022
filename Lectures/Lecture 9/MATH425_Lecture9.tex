\documentclass[12pt,a4paper]{article}
\usepackage[utf8]{inputenc}
\usepackage{amsmath}
\usepackage{enumitem}
\usepackage{amsfonts}
\usepackage{amssymb}
\usepackage{tikz}
\usepackage{amsmath}
\usepackage{amssymb}
\usepackage{pgfplots}
\usepackage{nccmath}
\usepackage{mathtools}
\usepackage{pgfplots}
\usepackage{mathtools,amssymb}
\usepackage{tikz}
\usepackage{xcolor}
\pgfplotsset{compat = newest}
\author{Chris Camano: ccamano@sfsu.edu}
\title{MATH 425 Lecture 9 }
\date{2/24/2022}
% Margins
\topmargin=-0.45in
\evensidemargin=0in
\oddsidemargin=0in
\textwidth=6.5in
\textheight=9.0in
\headsep=0.25in
%%%%%%Macros for 425%%%%%%%%%
\newcommand{\q}{\quad}
\newcommand{\tab}{\\\\}
\renewcommand{\labelenumi}{\alph{enumi})}
\newcommand{\sect}[1]{\section*{#1}}

\newcommand{\R}{\mathbb{R}}
\newcommand{\C}{\mathbb{C}}
\newcommand{\F}{\mathbb{F}}
\newcommand{\rtwo}{\mathbb{R}^2}
\newcommand{\mxn}{{mxn}}

\newcommand{\Axb}{\textbf{Ax=b} }
\newcommand{\Axz}{\textbf{Ax=0} }
\newcommand{\dim}{\text{dim}}
\newcommand{\lc}{linear combination }
%%%%%%%%%%%%%%%%%%%%%%%%%%%%%
\begin{document}
\maketitle
\sect{Linear Transformations}

Let U and V be vetor spaces over $\mathbb{F}( \R,\C,GF(2))$ let $ x,y \in U$ and $c\in \F$\tab
$T: U \mapsto V $ is a linear transformation $\iff$
  \begin{align*}
    T(x+y)=T(x)+T(y)\\
    T(cx)=cT(x)\\ \therefore\\
    T(cx+y)=cT(x)+T(y)
  \end{align*}
\textbf{A linear operator}\\
A linear operator is a transformation : \[
  T: U \mapsto U
\] this is to say linear transformation that takes a vector space to itself\tab
A transformation T is not linear if $T(\vec{0}_U) \neq \vec{0}_V$
\tab
\sect{Framing Linear Transformations}
 A linear transformation can be thought of as a function that maps some$x \in \F^n$ where $\F^n$ is the domain to some $y\in\F^m$ where $\F^m$ is the co-domain. The domain and co domain do not have to have different dimensions.\\
 \[
   \text{if } \text{dim}(U)< \text{dim}(V)
 \]
 The region a linear transformation maps the values of the domain to is called the range r where:
 \[
    r \subset \F^m
 \]
This is to say the range is a subset of the codomain of the linear transformation \tab
\textbf{Boateng description}\\
T(x) is the image of x under the action of T the set of all images T (x) is called the range of T.
\tab
\textbf{Additional properties of linear transformations}\\
All matrix-vector multiplications are linear transformations:
\[
  \text{Let} A \in \F^\mxn, x,y \in \F^n \land c \in \F
\]
\[
  A(cx+y) = cA(x)+A(y)
\]
All linear transformations on finite dimensional vector spaces
will alwways have a matrix representation
\tab
The standard matrix A for the linear transformation T: $\F^n \mapsto \F^m$ is completely determined by what id does to the columns of the nxn identity matrix $\mathbb{I}_n}$
\\
Proof:
\[
  \text {Let }x \in \F^n \rightarrow x=\matbb{I}=x\begin{bmatrix}
    |&|& &|\\
    e_1,&e_2,&..,&e_n\\
    |&|& &|\\
\end{bmatrix}
\]
Then:
\begin{align*}
  T(x)=t(x_1e_1+...+x_ne_n)\\
  =x_1T(e_1)+....+x_nT(e_n)\\
  \begin{bmatrix}
    |&|& &|\\
    T(e_1),&T(e_2),&..,&T(e_n)\\
    |&|& &|\\
\end{bmatrix}\begin{bmatrix}
x_1\\...\\x_n
\end{bmatrix}
\end{align*}

\end{tabular}
\end{document}
