\documentclass[12pt,a4paper]{article}
\usepackage[utf8]{inputenc}
\usepackage{amsmath}

\usepackage{amsfonts}
\usepackage{amssymb}
\usepackage{tikz}
\usepackage{amsmath}
\usepackage{amssymb}
\usepackage{pgfplots}
\usepackage{nccmath}
\usepackage{mathtools}
\usepackage{pgfplots}
\usepackage{mathtools,amssymb}
\usepackage{tikz}
\usepackage{xcolor}
\pgfplotsset{compat = newest}
\author{Chris Camano: ccamano@sfsu.edu}
\title{MATH425 Lecture 4 }
\date{2/3/2022}
% Margins
\topmargin=-0.45in
\evensidemargin=0in
\oddsidemargin=0in
\textwidth=6.5in
\textheight=9.0in
\headsep=0.25in
\newcommand{\q}{\quadd}
\renewcommand{\labelenumi}{\alph{enumi})}
\newcommand{\rtwo}{$\mathbb{R}^2$}
\newcommand{\C}{$\mathbb{C}$}

\begin{document}
\maketitle

\section{Vectors}
For this course we will be using a more abstract definition of a vector relating to a vector containing objects as opposed to raw numerical data. This decision is related to the selection of python versus MatLab. \\

Vectors can be expressed in many different forms, for example a 2-vector consists of two elements and can be represented on the cartesian plane as a scaling of the basis vectors from the origin.
\begin{align*}
  v=
  \begin{bmatrix}
            1 \\
            2 \\
  \end{bmatrix}\\
\end{align*}
These two entries in the vector are with respect to the vector space the vector originates from. This means that is $$ v \in \mathbb{R}^2$$ these are real numbers and if $$ v \in \mathbb{C} $$ one component is real and the other is complex

Vectors of the same vector space can be added component by component such that: \
\begin{align*}
  \begin{bmatrix}
            1 \\
            2 \\
  \end{bmatrix}
  +
  \begin{bmatrix}
            -1 \\
            2 \\
  \end{bmatrix}
  =
  \begin{bmatrix}
            0 \\
            4 \\
  \end{bmatrix}\\
\end{align*}
\begin{align*}
  \forall v,u,w \in \mathbb{F}^D\\
  \textbf{  associativity}\\\\
  (u+w)+v=u+(w+v)  \\\\
  \textbf{  commutativity}\\\\
  u+v=v+u \\
\end{align*}
We will define a vector in $\mathbb{F}^D$ as a function from D to $\mathbb{F}$. For this course the relevant fields include \rtwo,  $\mathbb{C}$, and GF2\\


\section{Vector representations}
One way to represent vectors is with a python list data structure. We will also represent vectors as the dictionary data structure where each index of the vector corresponds to a numerical key in the dictionary.  \\
Vectors will be created using the Vec class paramterized constructor. The arguments of the constructor will be the domain and co domain of the vector. Given a set of elements in the co domain of a vector you can expedite this process by writing

\[
  \textbf{v=Vec(\{i for i in range(len(C))\},C)}
\]
Where C is the co domain\\
\end{document}
