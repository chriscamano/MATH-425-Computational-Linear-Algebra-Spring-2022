\documentclass[12pt,a4paper]{article}
\usepackage[utf8]{inputenc}
\usepackage{amsmath}

\usepackage{amsfonts}
\usepackage{amssymb}
\usepackage{tikz}
\usepackage{amsmath}
\usepackage{amssymb}
\usepackage{pgfplots}
\usepackage{nccmath}
\usepackage{mathtools}
\usepackage{pgfplots}
\usepackage{mathtools,amssymb}
\usepackage{tikz}
\usepackage{xcolor}
\pgfplotsset{compat = newest}
\author{Chris Camano: ccamano@sfsu.edu}
\title{MATH425 Lecture 2 }
\date{2/1/2022}
% Margins
\topmargin=-0.45in
\evensidemargin=0in
\oddsidemargin=0in
\textwidth=6.5in
\textheight=9.0in
\headsep=0.25in
\newcommand{\q}{\quadd}
\renewcommand{\labelenumi}{\alph{enumi})}
\begin{document}
\maketitle

\section{Complex numbers - Plotting}\\
imaginary numbers can be plotted on an extension of $\mathbb{R}$ analagous to the cartesian plane in which an imaginary axis extends out from $\mathbb{R}$. \\\\
Complex numbers can also be plotted using polar coordinates with the length of r as the absolute value of some complex number Z\\
\[
  r=|z|=\sqrt{x^2+y^2}
\]
Also represented as :
\begin{align*}
  z=x+iy=r(cos(\theta)+isin(\theta))=re^{i\theta}
\end{align*}
\\\\
\subsection{Example}\\
Convert 1+i to a polar form\\
Convert $z=e^{i\pi/2}$to z=a+bi\\
Find e^{\pi i}+1

\begin{align*}
  1+i=\sqrt{2}e^{i\pi /4}=(\sqrt{2},\frac{\pi}{4})\\
  z=e^{i\pi/2}=cos(\frac{\pi}{2})+sin(\frac{\pi}{2})i\\
  e^{\pi i}+1=0
\end{align*}
\section{Rotating a complex numbers by $\tau$ radians}\\
Let f be the function that rotates $z=re^{i\theta}$ by $\tau$ counter clockwise. \\
\[
  f(z)= re^{i(\theta+\tau)}=ze^{i\tau}
\]
\end{document}
