\documentclass[12pt,a4paper]{article}
\usepackage[utf8]{inputenc}
\usepackage{amsmath}
\usepackage{amsfonts}
\usepackage{amssymb}
\usepackage{tikz}
\usepackage{amsmath}
\usepackage{amssymb}
\usepackage{nccmath}
\usepackage{mathtools}
\usepackage{pgfplots}
\usepackage{mathtools,amssymb}
\usepackage{tikz}
\usepackage{xcolor}
\pgfplotsset{compat=1.7}
\author{Chris Camano: ccamano@sfsu.edu}
\title{MATH425 Lecture 2}
\date{1/27/2022}
% Margins
\topmargin=-0.45in
\evensidemargin=0in
\oddsidemargin=0in
\textwidth=6.5in
\textheight=9.0in
\headsep=0.25in


\begin{document}
\maketitle
\section{The field}\\
A field $\mathbb{F}$ is a set containing at least two distinct elements, an additive identity or \{0\}. And a multiplicative Identity \{1\}, and the elements of the set that satisfy the following properties \\\\
\section{Properties of a Field}
$$
  \text{Commutativity:  } \alpha + \beta = \beta + \alpha\qquad  \forall \alpha\beta \in \mathbb{F} $$
  \\
  $$\text{Associativity:  } \alpha + (\beta+\lambda) = ()\alpha + \beta)+\lambda\qquad (\alpha\beta)\lambda=\alpha(\beta\lambda)\qquad \forall \alpha\beta \in \mathbb{F} $$
  \\
  $$\text{Identities:} \lambda + 0 = \lambda \text{ and } \lambda(1)=\lambda \quad \forall \lambda \in \mathbb{F}$$
  \\
  $$\text{Additive inverse: } \forall \alpha \in \mathbb{F} \exists \beta \in \mathbb{F} \text{ such that } \alpha + \beta=0$$
  \\
  $$\text{Multiplicative inverse:} \forall \alpha \in \mathbb{F} \text{ with} \alpha \neq 0,\exists \beta \in \mathbb{F} \text{ such that } \alpha\beta=1$$
  \\
  $$\text{Distributive property} \lambda(\alpha+\beta)=\lambda\alpha +\lambda\beta \quad \forall \lambda,\alpha,\beta \in \mathbb{F}
$$
\\
\textbf{Examples of Fields}\\
$\mathbb{R}$ set of real numbers\\
GF(2): Galois field of order 2. finite field: integers modulo 2 = \{0,2\}
\\\\
\textbf{Characteristics of GF2}\\
Since GF2 has two elemets it forms a 2 dimensions matri for addition with zeros on the principal diagonal and 1s on the other indicies.\\
\[
 \begin{bmatrix}
0 & 1 \\
1 & 0
\end{bmatrix}
\]
Multiplication for GF(2) also forms a 2x2 matrix as again, there are only two elements.
$$ \begin{bmatrix}
0 & 0 \\
0 & 1
\end{bmatrix}  $$
In GF(2) you take the modulous of anything larger than 2 becuase the set in constrained to two elements. \\

\section{Re-Introduction to Complex Numbers}\\
The complex plane is a field that consists of the set of complex numbers.
\\
\[
  \mathbb{C}=\{a+bi, a,b \in \mathbb{R}\}
\]
Where i= $\sqrt{-1}$. \\
In python j is used in place of i as a fragement of applications in electrical engineering.

\[
  z=a+bi= Re(z)+iIm(z)
\]
\begin{align*}
  \text{Re(z)=a, Im(z)=b}
\end{align*}


\end{document}
