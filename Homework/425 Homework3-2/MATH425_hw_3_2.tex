\documentclass[12pt]{article}
\usepackage[pdftex]{graphicx}
\usepackage{amsmath}
\usepackage{amssymb}
\pagestyle{empty}
\author{Chris Camano: ccamano@sfsu.edu}
\title{MATH425  Problem 9 submission}
\date{2/26/2022}

\topmargin -0.6in
\headsep 0.40in
\oddsidemargin 0.0in
\textheight 9.0in
\textwidth 6.5in

\newcommand{\econst}{\mathrm{e}}
\newcommand{\diff}{\mathrm{d}}
\newcommand{\dwrt}[1]{\frac{\diff}{\diff #1}}
%%%%%%Macros for 425%%%%%%%%%
\newcommand{\q}{\quad}
\newcommand{\tab}{\\\\}
\renewcommand{\labelenumi}{\alph{enumi})}
\newcommand{\sect}[1]{\section*{#1}}

\newcommand{\R}{\mathbb{R}}
\newcommand{\C}{\mathbb{C}}
\newcommand{\F}{\mathbb{F}}
\newcommand{\rtwo}{\mathbb{R}^2}
\newcommand{\mxn}{{mxn}}

\newcommand{\Axb}{\textbf{Ax=b} }
\newcommand{\Axz}{\textbf{Ax=0} }
\newcommand{\dim}{\text{dim}}
\newcommand{\lc}{linear combination }
%%%%%%%%%%%%%%%%%%%%%%%%%%%%%
\everymath={\displaystyle}


\begin{document}
\maketitle
\begin{proof}
  9.~  Find the $ 3 \times 3$ matrices that produce the described composite 2D transformations, using homogeneous coordinates. Apply the transformations to the {\bf letter N} data, ``letterN.pny" and submit the corresponding plots as well.
  \begin{itemize}
  \item[(a)] Translate by $(-2, 3)$, and then scale the $x$-coordinate by $0.8$ and the $y$-coordinate by $1.2$
  \[
      \begin{bmatrix}
        .8&0&0\\
        0&1.2&0\\
        0&0&1
      \end{bmatrix}\begin{bmatrix}
        1&0&-2\\
        0&1&3\\
        0&0&1
      \end{bmatrix}=\begin{bmatrix}
        .8&0&-1.6\\
        0&1.2&3.6\\
        0&0&1
      \end{bmatrix}
  \]
  \item[(b)] Rotate points $\frac{\pi}{6}$, and then reflect through the $x$-axis.
  For this problem since it is not specified I will be rotating counter clockwise
  \[
  \renewcommand{\arraystretch}{2.5}
    \begin{bmatrix}
    1&0&0\\
          0&-1&0\\
      0&0&1
    \end{bmatrix}\begin{bmatrix}
      \cos{\frac{\pi}{6}}&-\sin{\frac{\pi}{6}}&0\\
      \sin{\frac{\pi}{6}}& \cos{\frac{\pi}{6}}&0\\
      0&0&1
    \end{bmatrix}=\begin{bmatrix}
      \cos{\frac{\pi}{6}}&-\sin{\frac{\pi}{6}}&0\\
      -\sin{\frac{\pi}{6}}& -\cos{\frac{\pi}{6}}&0\\
      0&0&1
    \end{bmatrix}
  \]
  \end{itemize}
\end{proof}
\end{document}
