\documentclass[12pt,a4paper]{article}
\usepackage[utf8]{inputenc}
\usepackage{amsmath}

\usepackage{amsfonts}
\usepackage{amssymb}
\usepackage{tikz}
\usepackage{amsmath}
\usepackage{amssymb}
\usepackage{pgfplots}
\usepackage{nccmath}
\usepackage{mathtools}
\usepackage{pgfplots}
\usepackage{mathtools,amssymb}
\usepackage{tikz}
\usepackage{xcolor}
\pgfplotsset{compat = newest}
\author{Chris Camano: ccamano@sfsu.edu}
\title{MATH425 Homework  2 }
\date{2/3/2022}
% Margins
\topmargin=-0.45in
\evensidemargin=0in
\oddsidemargin=0in
\textwidth=6.5in
\textheight=9.0in
\headsep=0.25in
\newcommand{\q}{\quad}
\renewcommand{\labelenumi}{\alph{enumi})}
\newcommand{\R}{\mathbb{R}}
\newcommand{\rtwo}{$\mathbb{R}^2$}
\newcommand{\C}{$\mathbb{C}$}

\begin{document}
\maketitle
\noindent
Math 425 \qquad
Applied and Computational Linear Algebra \qquad
Spring 2022
\vskip 5pt
\noindent
\section*{Problem 1}
\begin{proof}
  : List five vectors in Span $\{v_1, v_2\}$. For each vector, show the weights on $v_1, v_2$ used to
  generate the vector and list the three entries of the vector. Do not make a sketch.
  \[
    v_1=\begin{bmatrix}
      3\\
      0\\
      2\\
  \end{bmatrix}
  v_2=\begin{bmatrix}
    -2\\
    0\\
    3\\
  \end{bmatrix}
  \]
  \begin{align*}
    \text{let } w=4v_1+3v_2\\
    w=  4\begin{bmatrix}
        3\\
        0\\
        2\\
    \end{bmatrix}
    +3\begin{bmatrix}
      -2\\
      0\\
      3\\
    \end{bmatrix}
    =\begin{bmatrix}
      -6\\
      0\\
      17\\
    \end{bmatrix}
  \end{align*}
  \begin{align*}
    \text{let } u=8v_1-2v_2\\
    u=  8\begin{bmatrix}
        3\\
        0\\
        2\\
    \end{bmatrix}
    -2\begin{bmatrix}
      -2\\
      0\\
      3\\
    \end{bmatrix}
    =\begin{bmatrix}
      28\\
      0\\
      10\\
    \end{bmatrix}
  \end{align*}
  \[
    u,w \in \textbf{Span}\{v_1,v_2\} \text{ as} \q \forall u_i \in u, w_i \in w: \quad u_i,w_i=\lambda v_{1_i}+ \mu v_{2_i}
  \]
  Other Examples
  \begin{align*}
    \text{let } f=3v_1-v_2\\
    f=  3\begin{bmatrix}
        3\\
        0\\
        2\\
    \end{bmatrix}
    -\begin{bmatrix}
      -2\\
      0\\
      3\\
    \end{bmatrix}
    =\begin{bmatrix}
      11\\
      0\\
      3\\
    \end{bmatrix}
  \end{align*}
  \begin{align*}
    \text{let } h=2v_1+v_2\\
    h=  2\begin{bmatrix}
        3\\
        0\\
        2\\
    \end{bmatrix}
    +\begin{bmatrix}
      -2\\
      0\\
      3\\
    \end{bmatrix}
    =\begin{bmatrix}
      4\\
      0\\
      7\\
    \end{bmatrix}
  \end{align*}
  \begin{align*}
    \text{let } y=v_1-v_2\\
    y=  4\begin{bmatrix}
        3\\
        0\\
        2\\
    \end{bmatrix}
    -\begin{bmatrix}
      -2\\
      0\\
      3\\
    \end{bmatrix}
    =\begin{bmatrix}
      5\\
      0\\
      -1\\
    \end{bmatrix}
  \end{align*}
\end{proof}
\section*{Problem 2}
\begin{proof}
  Decide whether the following sets of vectors are linearly independent or linearly dependent.
  Give reasons for your choices.\\
  \textbf{Problem a}\\
  \[
    \Bigg\{
    \begin{bmatrix}
        1\\
        0\\
        3\\
    \end{bmatrix}
    ,\begin{bmatrix}
      -3\\
      2\\
      -7\\
    \end{bmatrix}
    ,\begin{bmatrix}
      2\\
      -11\\
      -8\\
    \end{bmatrix}
    \Bigg\}
  \]
  \[
    A=\begin{bmatrix}
      1 & -3 & 2 \\
      0 & 2 & -11 \\
      3 & -7 & -8
    \end{bmatrix}
    \sim
    \mathbb{I}^3\therefore
  \]
   the columns of A are linearly independent by the invertible matrix theorem\\
  \textbf{Problem b}\\
  \[
    \Bigg\{
    \begin{bmatrix}
        -3\\
        2\\
        -7\\
    \end{bmatrix}
    ,\begin{bmatrix}
      2\\
      -11\\
      -8\\
    \end{bmatrix}
    \Bigg\}
  \]
  \[
    A=\begin{bmatrix}
      -3 &2 \\
      2& -11
    \\ -7 & -8
  \end{bmatrix}\sim \begin{bmatrix}
    1 &0 \\
    0& 1
  \\ 0 & 0
  \end{bmatrix}
  \]

  The matrix A formed by the vectors in the set given create a matrix that only has the trivial solution therfore this system is linearly indepdent. \\
  \textbf{Problem c}\\
  \[
    \Bigg\{
    \begin{bmatrix}
        1\\
        0\\
        3\\
    \end{bmatrix}
    ,\begin{bmatrix}
      -3\\
      2\\
      -7\\
    \end{bmatrix}
    ,\begin{bmatrix}
      2\\
      -11\\
      -8\\
    \end{bmatrix},
    \begin{bmatrix}
      9\\
      12\\
      13\\
    \end{bmatrix}
    \Bigg\}
  \]
  \[
    A=\begin{bmatrix}
       1& -3 & 2 & 9 \\
       0 & 2 & -11 & 12\\
       3& -7 & -8 & 13
  \end{bmatrix}\\
  A \in \mathbb{R}^{3x4} \therefore
  \]
  A contains more vectors than there are basis vectors in $\mathbb{R}^3$ therefore this set of vectors must be linearly dependent\\
  \textbf{Problem d}\\
  \[
    \Bigg\{
    \begin{bmatrix}
        1\\
        4\\
        9\\
        10
    \end{bmatrix}
    ,\begin{bmatrix}
      0\\
      0\\
      0\\
      0\\
    \end{bmatrix}
    ,\begin{bmatrix}
      5\\
      7\\
      1\\
      0\\
    \end{bmatrix}
    \Bigg\}
  \]
  Any set containing the zero vector is linearly dependent as setting the zero vector with weight 1 and all other vectors with weight 0 is a solution to a system.
\end{proof}\\
\section*{Problem 3}
\begin{proof}
  \textbf{Problem a }:\\
  \[
  A=\begin{bmatrix}
    -4&-3&0&0\\
    0&-1&4&0\\
    1&0&3&0\\
    5&4&6&0\\
  \end{bmatrix}
  \sim
  A=\begin{bmatrix}
    1&0&0&0\\
    0&1&0&0\\
    0&0&1&0\\
    0&0&0&0\\
  \end{bmatrix}
  \]
  The matrix A can be reduced to a form such that there is a pivot in every column meaning we only have the trivial solution for the vectors in the column space of A. Therefore it is linearly indepdent\\
  \textbf{Problem B}\\
  Given the matrix
  \[

    A=\begin{bmatrix}
      1& -3&3&-2\\
      -3&7&-1&2\\
      0&1&-4&3\\
  \end{bmatrix}
  \]
  It is clear that $A \in \R^{3x4}$ meaning that A contains more vectors than there are basis vectors in $\R^3$ therefore the set must be linearly dependent. This can be verified through reduction:
  \[
  A=\begin{bmatrix}
    1& -3&3&-2&0\\
    -3&7&-1&2&0\\
    0&1&-4&3&0\\
  \end{bmatrix} \sim
  \begin{bmatrix}
    1& 0&-9&0&0\\
    0&1&-4&0&0\\
    0&0&0&1&0\\
  \end{bmatrix}
    \]
    Where $x_3$ is free
\end{proof}\\
\section*{Problem 4}
\begin{proof}
  \textbf{a).}
  Let:
  \[
    v_1=\begin{bmatrix}
      1\\-5\\-3
    \end{bmatrix},
    v_2=\begin{bmatrix}
      -2\\10\\6
  \end{bmatrix}
  \text{and }v_3=\begin{bmatrix}
    2\\-9\\h
  \end{bmatrix}
  \]
  If $v_3 \in \text{Span}\{v_1,v_2\}$ then $\exists c_1,c_2, (c_1,c_2 \neq 0) | v_3=c_1v_1+c_2v_2$
  \[
    \begin{bmatrix}
      1 & -2 & 2 \\-5 & 10 & -9\\ -3 & 6 & h
    \end{bmatrix}
    \sim \begin{bmatrix}-5&10&-9\\ 0&0&\frac{1}{5}\\ 0&0&\frac{5h+27}{5}\end{bmatrix}
  \]
  There is no pivot in column two of the system which means that the system is inconsistent, because of this h does not matter as there are no solutions to the system.\\
  \textbf{b).}
  For what values of h is $\{v_1,v_2,v_3\}$ linearly dependent?\\
  \[
  \begin{bmatrix}
    1 & -2 & 2&0 \\-5 & 10 & -9&0\\ -3 & 6 & h&0
  \end{bmatrix}
  \sim
  \begin{bmatrix}1&-2&0&0\\ 0&0&1&0\\ 0&0&0&0\end{bmatrix}
    \]
  since the matrix is not full rank this means that for all values h we  obtain a linearly dependent system.
\end{proof}\\
\section*{Problem 5}
\begin{proof}
  \[
  \begin{bmatrix}
    4&1&6&0\\-7&5&3&0\\9&-3&3&0
  \end{bmatrix}
  \sim
  \begin{bmatrix}
    1&0&1&0\\0&1&2&0\\0&0&0&0
  \end{bmatrix}
  \]
  \[
    v_1+2v_2-v_3=0 \quad \therefore
  \begin{bmatrix}
      -1\\-2\\1
  \end{bmatrix}
  \]
  is a solution for $v_3$. All vectors that scale this solution are also solutions for the problem. Solving the system here reveals that $x_3$ is a free variable.
\end{proof}\\
\section*{Problem 6}
\begin{proof}
  Could a set of three vectors in $\R^4$ span all of $\R^4$? What about n vectors in $\R^m$ given $n<m$.\\\\
  a set of three vectors in $\R^4$ is not capable of spanning all of $\R^4$ as the matrix formed by the set is not row reducible to to the identity matrix. This is to say that there exist vectors in $\R^4$ not spanned by our set of three vectors. the subspace spanned by the three vectors is three dimensional due to the fact that the matrix can only be reduced to three basis vectors. In order to span all of $\R^4$ you would need to have a full rank matrix for that vector space:
  \begin{align*}
    \text{Span}\{v_1,v_2,v_3\} \in \R^3\\
    R^3 \subset \R^4, \therefore\\
      \text{Span}\{v_1,v_2,v_3\}\subset \R^4
  \end{align*}
  This of course generalizes as it builds up the definition of vectorspace dimensionality: For all vector spaces the dimension of that vector space is defined by the number of vectors in a basis for V. This statement means that the number of basis vectors in a basis for V is the dimension for that vectorspace. Here given $n<m$ there is no way that that the n vectors can span the m dimensional Vector space becuase n does not equal the number of vectors in the basis for V
\end{proof}\\
\section*{Problem 7}
\begin{proof}
  \[
    \text{Let } u = \begin{bmatrix}
    1\\-3\\2
  \end{bmatrix}\text{ and } A=\begin{bmatrix}
    5 & 8 & 7  \\
    0 & 1 & -1\\ 1 & 3 & 0
\end{bmatrix}
  \text{ is } u \in \text{Span}\{A\}
    \]
    \[
    \begin{bmatrix}
      5 & 8 & 7&0  \\
      0 & 1 & -1&0\\ 1 & 3 & 0&0
    \end{bmatrix}\sim \begin{bmatrix}
      1 & 0 & 3&0  \\
      0 & 1 & -1&0\\ 0 &0 & 0&0
    \end{bmatrix}
    \]
    Which tells us that this matrix A does not span $\R^3$
    $$
    \begin{bmatrix}
      5 & 8 & 7&1  \\
      0 & 1 & -1&-3\\ 1 & 3 & 0&2
    \end{bmatrix}\sim \begin{bmatrix}
      1 & 0 & 3&0  \\
      0 & 1 & -1&0\\ 0 &0 & 0&1
    \end{bmatrix}
    $$
    which mean the system is inconsistent when solving for u so there do does not exist a linear combination of the columns of A such that the sum of the weighted vectors is equal to the vector u
\end{proof}
\section*{Problem 8}
\begin{proof}
    \[
      \text{ Let } A =\begin{bmatrix}
        1 & -3 & -4 \\
        -3 & 2 & 6 \\
        5 & -1 & -8
    \end{bmatrix}
    \text{ and } b= \begin{bmatrix}
      b_1\\b_2\\b_3
  \end{bmatrix}
    \]
    Show that \textbf{Ax=b} does not have a solution $\forall$ b and describe the set of all b for which \textbf{Ax=b} does have a solution
    \[
    \begin{bmatrix}
      1 & -3 & -4 \\
      -3 & 2 & 6 \\
      5 & -1 & -8
  \end{bmatrix} \sim
  \renewcommand\arraystretch{1.25}
  \begin{bmatrix}
    1 & 0 & \frac{-10}{7} \\
    0 & 1 & \frac{6}{7} \\
    0 & 0 & 0
  \end{bmatrix}
    \]
    This matrix is not row reducible to the identity matrix for $\R^3$ therefore it does not span all of $\R^3$ using this information we can conclude that there are some vectors b that lay outside of the span of the columns of A. These vectors are vectors that cannot be described as a linear combination of the columns of A. To find a vector that lays outside of the span of A all we have to do is describe a vector of the form :
    \[
      \begin{bmatrix}
        b_1\\b_2\\k
      \end{bmatrix},
      k \in \mathbb{R}
    \]
    As there is no solution for the system if there is a z component. the fact that the matrix is not row reducible to a basis for $R^3$ means that any vector with a z component exists outside of the span of A. Here is an example of a vector \textbf{b} such that \textbf{Ax \neq b}\\
    \[
      b=\begin{bmatrix}
        1\\1\\1
      \end{bmatrix},
      \]
      \[
    \begin{bmatrix}
      1 & -3 & -4 \\
      -3 & 2 & 6 \\
      5 & -1 & -8
    \end{bmatrix}
    \begin{bmatrix}
        x_1\\x_2\\x_3
      \end{bmatrix}=
      b=\begin{bmatrix}
        1\\1\\1
      \end{bmatrix},
    \]
    \[
    \begin{bmatrix}
      1 & -3 & -4 &1\\
      -3 & 2 & 6 &1\\
      5 & -1 & -8 &1
  \end{bmatrix} \sim
  \begin{bmatrix}
    1 & 0 & \frac{-10}{7} & 0 \\
    0 & 1 & \frac{6}{7}& 0 \\
    0 & 0 & 0& 1
  \end{bmatrix}
    \]
    Thus b is not included in the subspace spanned by A
\end{proof}
\section*{Problem 9}
\begin{proof}
  \[
    \text{ Let }B=\begin{bmatrix}
      1& 3 & -2 & 2 \\
      0 & 1 & 1 & -5\\
       1 & 2 &-3 & 7 \\
       -2 & -8 & 2 & -1
  \end{bmatrix}
  \]
  is \textbf{Span\{B\}} $\R^4$? Does \textbf{Bx=y} have a solution $ \forall y \in \R^4$?
  \[
  \begin{bmatrix}
    1& 3 & -2 & 2 \\
    0 & 1 & 1 & -5\\
     1 & 2 &-3 & 7 \\
     -2 & -8 & 2 & -1
  \end{bmatrix} \sim
  \begin{bmatrix}
    1& 0 & -5 & 0 \\
    0 & 1 & 1 & 0\\
     0 & 0 &0 & 1 \\
     0 & 0 & 0 & 0
  \end{bmatrix}
  \]
  This matrix is inconsistent and thus does not span all of $\R^4$
  A counter example to the statement $\forall y \in \R^4, \exists x | \text{Bx=y}$ is:
  \[
  y=
    \begin{bmatrix}
      1 \\ 1\\ 1 \\ 1
    \end{bmatrix}
  \]
  \[
  \begin{bmatrix}
    1& 3 & -2 & 2 &1\\
    0 & 1 & 1 & -5&1\\
     1 & 2 &-3 & 7 &1\\
     -2 & -8 & 2 & -1 &1
  \end{bmatrix} \sim
  \begin{bmatrix}
    1& 0 & -5 & 0 &0\\
    0 & 1 & 1 & 0&0\\
    0 & 0 &0 & 1 &0\\
    0 & 0 & 0 & 0 &1
  \end{bmatrix}
  \]
  The system is inconsistent therefore it is a vector that is not does not satisfy \text{Bx=y}
\end{proof}
\end{document}
