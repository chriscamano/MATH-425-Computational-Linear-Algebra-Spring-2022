\documentclass[12pt,a4paper]{article}
\usepackage[utf8]{inputenc}
\usepackage{amsmath}
\usepackage{enumitem}
\usepackage{amsfonts}
\usepackage{amssymb}
\usepackage{tikz}
\usepackage{amsmath}
\usepackage{amssymb}
\usepackage{pgfplots}
\usepackage{nccmath}
\usepackage{mathtools}
\usepackage{pgfplots}
\usepackage{mathtools,amssymb}
\usepackage{tikz}
\usepackage{xcolor}
\pgfplotsset{compat = newest}
\author{Chris Camano: ccamano@sfsu.edu}
\title{MATH425 Homework 1}
\date{2/5/2022}
% Margins
\topmargin=-0.45in
\evensidemargin=0in
\oddsidemargin=0in
\textwidth=6.5in
\textheight=9.0in
\headsep=0.25in
\newcommand{\q}{\quadd}
\renewcommand{\labelenumi}{\alph{enumi})}
\begin{document}
\maketitle
\section{Problem 1:}
Solve the following system:
\begin{align*}
  u + v + w = -2\\
  3u + 3v - w = 6\\
  u - v + w =-1
\end{align*}
expressed as an augmented matrix this system of linear equations is:
\[
 \begin{bmatrix}
1 & 1 & 1 & -2 \\
3 & 3 & -1 & 6 \\
1 & -1 & 1 & -1
\end{bmatrix}
\]
the following matrix when reduced to row echelon form reveals the solution to the system of linear equations.
\begin{align*}
  \begin{bmatrix}
 1 & 1 & 1 & -2 \\
 3 & 3 & -1 & 6 \\
 1 & -1 & 1 & -1
 \end{bmatrix} \sim
  \begin{bmatrix}
1 & 0 & 0 & \frac{3}{2} \\
0 & 1 & 0 & -\frac{1}{2} \\
0 & 0 & 1 & -3
\end{bmatrix}
\end{align*}
implying that the solution to the system is that:
\begin{align*}
  u=\frac{3}{2}\\
    v=-\frac{1}{2}\\
   w=-3
\end{align*}
\section{Problem 2:}\\
Choose h and k such that the system below has (a) no solution, (b) a unique solution,
and (c) many solutions. Give separate answers for each part.
\begin{align*}
  x_1 + 3x_2 = 2\\
3x_1 + hx_2 = k\\
\begin{bmatrix}
1 & 3 & 2 \\
3 & h & k \\
\end{bmatrix}
\end{align*}
\begin{enumerate}[label=(\alph*)]
\item No  solutions:
\begin{align*}
  \textbf{R2=-3R1+R2}\\
  \begin{bmatrix}
  1 & 3 & 2 \\
  3 & h & k \\
  \end{bmatrix}  =
  \begin{bmatrix}
  1 & 3 & 2 \\
  0 & h-9 & k-6 \\
  \end{bmatrix}
\end{align*}
Thus the system is inconsistent when $\textbf{h=9}$ and $k \neq 6$
\item A unique solution
\begin{center}
  by the logic of the reduction in part a the system is consistent for all values of h not equal to 9 as there will be a pivot in row 2 if this is the case.
\end{center}
\item Infinite solutions
\begin{center}
  By the logic of the reduction in part a the system is consistent with Infinite solutions when h=9 and k =6.
\end{center}
\end{enumerate}

\section{Problem 3:}\\
Consider the system below:
\begin{align*}
  4x_1 + x_2 + 3x_3 = 9\\
  x_1 - 7x_2 - 2x_3 = 12\\
  8x_1 + 6x_2 - 5x_3 =  15\\
  \begin{bmatrix}
 4 & 1 & 3 & 9 \\
 1 & -7 & -2 & 12 \\
 8 & 6 & -5 & 15
 \end{bmatrix}
\end{align*}
  \begin{enumerate}[label=(\alph*)]
    \item  Column-space view: Find the vectors $v_1,v_2,v_3$and write the system as a vector equation:
      \begin{align*}
        x_1v_1+x_2v_2+x_3v_3=
         \begin{bmatrix}
          9 \\
          12 \\
          15
        \end{bmatrix}\\
        x_1
        \begin{bmatrix}
         1 \\
         -7 \\
         6
       \end{bmatrix}
       +x_2
       \begin{bmatrix}
        3 \\
        -2 \\
        -5
      \end{bmatrix}
      +x_3\begin{bmatrix}
       3 \\
       -2 \\
       -5
     \end{bmatrix}
     =
      \begin{bmatrix}
       9 \\
       12 \\
       15
     \end{bmatrix}\\
 \end{align*}
 \item Row-space view: Find the vectors $w_1,w_2,w_3$ and x such that the system is equivalent to:
     \begin{align*}
       w_1\cdot x=  9\\
       w_2 \cdot x=  12\\
       w_3\cdot x=  15
     \end{align*}
     \begin{align*}
       w_1 = [4 , 1 , 3]\\
       w_2 = [1 , -7 , -2]\\
       w_3 = [8 , 6 , -5]
     \end{align*}
     \begin{align*}
      [4 , 1 , 3] \cdot
      \begin{bmatrix}
       x_1 \\
       x_2 \\
       x_2
     \end{bmatrix}=9 \\
      [1 , -7 , -2] \cdot
      \begin{bmatrix}
       x_1 \\
       x_2 \\
       x_2
     \end{bmatrix}=12\\
      [8 , 6 , -5]\cdot
      \begin{bmatrix}
       x_1 \\
       x_2 \\
       x_2
     \end{bmatrix}=15
     \end{align*}
    Let x be the following solution to the system to validate this claim. Solution obtained using row operations.
    \renewcommand\arraystretch{2}
    $\begin{bmatrix}
     \frac{328}{121}\\
     \frac{-14}{11}\\
     \frac{-23}{121}
    \end{bmatrix}$
    \begin{align*}
     [4 , 1 , 3] \cdot
     \begin{bmatrix}
      \frac{328}{121}\\
      \frac{-14}{11}\\
      \frac{-23}{121}
     \end{bmatrix}=9 \\
     [1 , -7 , -2] \cdot
     \begin{bmatrix}
      \frac{328}{121}\\
      \frac{-14}{11}\\
      \frac{-23}{121}
     \end{bmatrix}=12\\
     [8 , 6 , -5]\cdot
     \begin{bmatrix}
      \frac{328}{121}\\
      \frac{-14}{11}\\
      \frac{-23}{121}
     \end{bmatrix}=15
    \end{align*}
\end{enumerate}
\section{Problem 4: }
Determine if $\textbf{b}$ is a linear combination of the vectors formed from the columns of the matrix $\textbf{A}$ .
\begin{align*}
  \textbf{A}=
    \begin{bmatrix}
   1 & -2 & -6  \\
   0 & 3 & 6  \\
   1 & -2 & 5
   \end{bmatrix},
   \textbf{b}=\begin{bmatrix}
  11  \\
  -5   \\
  9
  \end{bmatrix}
\end{align*}
\begin{align*}
  \begin{bmatrix}
 1 & -2 & -6 & 11\\
 0 & 3 & 6 & -5\\
 1 & -2 & 5& 9
 \end{bmatrix} \sim
   \renewcommand\arraystretch{2}
 \begin{bmatrix}
1 & 0 & 0 & \frac{241}{33}\\
0 & 1 & 0 & \frac{-43}{33}\\
0 & 0 & 1 & \frac{-2}{11}
\end{bmatrix}
\end{align*}
A unique solution exists for the given existence problem therefore yes $\textbf{b}$ is a linear combinations of the vectors in \textbf{A}
\section{Problem 5}
Let f(z) = az + b where $z \in \mathbb{C}$. Find a and b if f(z) translates z one unit up and one
unit to the right, rotates the result by $\frac{\pi}{2}$
clockwise and scales the resulting complex number
by 2.\\\\
Since the input value z is rotate around clockwise b y$\frac{\pi}{2}$ this means that part of the coefficient must be $e^{-\frac{\pi}{2}i}$ so to start we have:
\[
  f(z)=e^{-\frac{\pi}{2}i}z+b
\]
Next we must scale the complex number by 2 modifying the function to be
\[
  f(z)=2e^{-\frac{\pi}{2}i}z+b
\]
Finally to move each output up one and one to the right we must add $1+i$ to the result for a final description of
\[
  f(z)=2e^{-\frac{\pi}{2}i}z+(1+i))
\]
\end{document}
