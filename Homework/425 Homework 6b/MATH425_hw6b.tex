\documentclass[12pt]{article}
\usepackage[pdftex]{graphicx}
\usepackage{amsmath}
\usepackage{amssymb}
\pagestyle{empty}
\author{Chris Camano: ccamano@sfsu.edu}
\title{MATH 425  Homework 6b }
\date{4/25/2022}

\topmargin -0.6in
\headsep 0.40in
\oddsidemargin 0.0in
\textheight 9.0in
\textwidth 6.5in

\newcommand{\econst}{\mathrm{e}}
\newcommand{\diff}{\mathrm{d}}
\newcommand{\dwrt}[1]{\frac{\diff}{\diff #1}}
%%%%%%Macros for 425%%%%%%%%%%%%%%%%%%%
\newcommand{\q}{\quad}
\newcommand{\tab}{\\\\}
\renewcommand{\labelenumi}{\alph{enumi})}
\newcommand{\sect}[1]{\section*{#1}}

%%%%%%Vector Spaces%%%%%%%%%%%%%%%%%%%
\newcommand{\R}{\mathbb{R}}
\newcommand{\C}{\mathbb{C}}
\newcommand{\F}{\mathbb{F}}
\newcommand{\rtwo}{\mathbb{R}^2}
\newcommand{\mxn}{{mxn}}

%%%%%%Sets and common phrases%%%%%%%%%
\newcommand{\Axb}{\textbf{Ax=b} }
\newcommand{\Axz}{\textbf{Ax=0} }
\newcommand{\dim}{\text{dim}}
\newcommand{\lc}{linear combination }
\newcommand{\let}{\text{Let }}
\newcommand{\tf}{\therefore}
%%%%%%%%%%%%%%%%%%%%%%%%%%%%%%%%%%%%%%%
\everymath={\displaystyle}


\begin{document}
\maketitle
\sect{Problem 1}
\begin{proof}
  Find the singular values of #\begin{bmatrix}
    -5&0\\0&0
\end{bmatrix}
\\\\
We begin by computing $A^TA$ which is as follows:
\[
\begin{bmatrix}
  -5&0\\0&0
\end{bmatrix}^T\begin{bmatrix}
  -5&0\\0&0
\end{bmatrix}=\begin{bmatrix}
  25&0\\0&0
\end{bmatrix}
\]
The characteristic polynomial for  $A^TA$ is then : $\lambda^2-25\lambda$ meaning our eigenvalues are 25 and 0. From here finding the singular values amounts to taking the square root of the eigen values yeilding: 5 and 0 as the final answer.
\end{proof}
\sect{Problem 2}
\begin{proof}
\begin{itemize}
  \item Since we have only two nonzero singular values for this decomposition this means that the rank of the matrix is 2.
  \item Since the rank of the matrix is equal to 2 then the basis for ColA is equivilant to the first 2 vectors of U. . A basis for NulA is equivilant to the $v_{r+1}=v_n#$ columns of V. In this case this means that a basis for NulA is the last 2 columns of V in our decomposition.
\end{itemize}
\end{proof}
\sect{Problem 3}
\begin{proof}
Let \[
  A=U\Sigma V^T
\]
Then:
\begin{align*}
  &A^{-1}=(U\Sigma V^T)^{-1}\\
  &A^{-1}=V^{T^{-1}}\Sigma ^{-1}U^{-1}\\
  &\text{U and V are orthogonal so U or V tranpose is equal to the inverse:}\\
  &A^{-1}=V^{T^T}\Sigma ^{-1}U^T\\
  &A^{-1}=V\Sigma ^{-1}U^T\\
\end{align*}
\end{proof}
\sect{Problem 4}
\begin{proof}
\begin{align*}
  &|det(A)|=|det(U\Sigma V^T)|\\
  &|det(A)|=|det(U)det(\Sigma)det(V^T)|\\
  &|det(A)|=|det(U)\Big(\prod_{i=1}^r\sigma_i\Big)det(V^T)|\\
\end{align*}
\textbf{lemma: determinant of an orthogonal matrix is 1 or -1}\\
Since the matrix is orthogonal $AA^T=\mathbb{I}_n$ thus \\
\[
  det(AA^T)=det(\mathbb{I}_n)
\]
\[
  det(AA^T)=1
\]the determinant is a linear transformation so we can expand the left hand side.
\[
    det(A)det(A^T)=1
\]
det(A)=det($A^T$)so we can reexpress this as:
\[
    det(A)^2=1
\]
meaning that \[
    det(A)=\pm1
\]
for any orthogonal matrix.
returning to our proof
\begin{align*}
  &|det(A)|=|det(U)det(\Sigma)det(V^T)|\\
  &|det(A)|=|\pm1\Big(\prod_{i=1}^r\sigma_i\Big)\pm1|\\
  &|det(A)|=|\pm1\Big(\prod_{i=1}^r\sigma_i\Big)|\\
  &|det(A)|=\prod_{i=1}^r\sigma_i\\
\end{align*}
\blacksquare
\end{proof}
\sect{Problem 5}
\begin{proof}
  We first start by computing the singular value decomposition of A in a step towards a pseudoinverse description of the minimal solution. \\
  *Values computed with numpy\\
  \[
    AA^t=\begin{bmatrix}
      2&2&1&1\\2&2&1&1\\1&1&2&2\\1&1&2&2
  \end{bmatrix}
  \]The eigenvalues of this matrix are \[
    \lambda_1=6\quad\lambda_2=2\quad\lambda_3=0\quad\lambda_4=0
  \]
  Thus D is equal to:
  \[
    \begin{bmatrix}
    6&0&0\\0&2&0\\0&0&0
    \end{bmatrix}
  \]
  computing relevant eigenvectors:
  \[
    v_1=\begin{bmatrix}
      1\\1\\1\\1
  \end{bmatrix}\quad v_2=\begin{bmatrix}
    -1\\-1\\1\\1
\end{bmatrix}
\quad v_3=\begin{bmatrix}
  -1\\1\\0\\0
\end{bmatrix}
\quad v_4=\begin{bmatrix}
  0\\0\\-1\\1
\end{bmatrix}
  \]
  Normalizing we form the columns of U as:
  \[

    U=\begin{bmatrix}
      \dfrac{1}{2}&-\dfrac{1}{2}&-\dfrac{1}{\sqrt{2}}&0\\
        \dfrac{1}{2}&-\dfrac{1}{2}&\dfrac{1}{\sqrt{2}}&0\\
          \dfrac{1}{2}&\dfrac{1}{2}&0&-\dfrac{1}{\sqrt{2}}\\
          \dfrac{1}{2}&\dfrac{1}{2}&0&\dfrac{1}{\sqrt{2}}\\
  \end{bmatrix}
  \]
  For the columns of V we compute $$v_i=\frac{1}{\sigma_i}A^Tu_i$$ yeilding:
  \[
    v_1=\begin{bmatrix}
      \frac{2}{\sqrt{6}}\\\frac{1}{\sqrt{6}}\\\frac{1}{\sqrt{6}}
  \end{bmatrix}
  \quad v_2=\begin{bmatrix}
    0\\\frac{-1}{\sqrt{2}}\\\frac{1}{\sqrt{2}}
\end{bmatrix}
  \]
  We are missing an additional vector for V so we will select a linearly independent one In this case one can be found that is orthogonal by solving $[v_1,v_2]^Tx=0$ (the nul space):
  \[
    v_3=\begin{bmatrix}
      -1\\1\\1
  \end{bmatrix}
  \]
  Meaning that
  \[
    V=\begin{bmatrix}
      \frac{2}{\sqrt{6}}&0&-1\\
      \frac{1}{\sqrt{6}}&\frac{-1}{\sqrt{2}}&1\\
      \frac{1}{\sqrt{6}}&\frac{1}{\sqrt{2}}&1
  \end{bmatrix}
  \]
  \[
    A=\begin{bmatrix}
      \dfrac{1}{2}&-\dfrac{1}{2}&-\dfrac{1}{\sqrt{2}}&0\\
        \dfrac{1}{2}&-\dfrac{1}{2}&\dfrac{1}{\sqrt{2}}&0\\
          \dfrac{1}{2}&\dfrac{1}{2}&0&-\dfrac{1}{\sqrt{2}}\\
          \dfrac{1}{2}&\dfrac{1}{2}&0&\dfrac{1}{\sqrt{2}}\\
  \end{bmatrix}\begin{bmatrix}
    \sqrt{6}&0&0\\0&\sqrt{2}&0\\0&0&0\\0&0&0
\end{bmatrix}\begin{bmatrix}
  \frac{2}{\sqrt{6}}&0&-1\\
  \frac{1}{\sqrt{6}}&\frac{-1}{\sqrt{2}}&1\\
  \frac{1}{\sqrt{6}}&\frac{1}{\sqrt{2}}&1
\end{bmatrix}^T
  \]
  \[
    D=\begin{bmatrix}
      \sqrt{6}&0\\0&\sqrt{2}
  \end{bmatrix}
  \]
  Since the rank of the matrix of 2:
  \[
    U_r=\begin{bmatrix}
    \dfrac{1}{2}&-\dfrac{1}{2}\\
      \dfrac{1}{2}&-\dfrac{1}{2}\\
        \dfrac{1}{2}&\dfrac{1}{2}\\
        \dfrac{1}{2}&\dfrac{1}{2}\\
  \end{bmatrix}
  \quad V_r=\begin{bmatrix}
  \frac{2}{\sqrt{6}}&0\\
  \frac{1}{\sqrt{6}}&\frac{-1}{\sqrt{2}}\\
  \frac{1}{\sqrt{6}}&\frac{1}{\sqrt{2}}
\end{bmatrix}
  \]
  \[
    \hat{x}=V_rD^-1U_r^Tb
  \]
  \[
    \hat{x}=\begin{bmatrix}
    \frac{2}{\sqrt{6}}&0\\
    \frac{1}{\sqrt{6}}&\frac{-1}{\sqrt{2}}\\
    \frac{1}{\sqrt{6}}&\frac{1}{\sqrt{2}}
  \end{bmatrix}\begin{bmatrix}
    \sqrt{6}&0\\0&\sqrt{2}
\end{bmatrix}^{-1}\begin{bmatrix}
\dfrac{1}{2}&-\dfrac{1}{2}\\
  \dfrac{1}{2}&-\dfrac{1}{2}\\
    \dfrac{1}{2}&\dfrac{1}{2}\\
    \dfrac{1}{2}&\dfrac{1}{2}\\
\end{bmatrix}^T\begin{bmatrix}
  1\\3\\8\\2
\end{bmatrix}
  \]
  \renewcommand*{\arraystretch}{2}
  \[
    \hat{x}=\begin{bmatrix}
      \dfrac{7}{3}\\\dfrac{-1}{3}\\\dfrac{8}{3}
  \end{bmatrix}
  \]
\end{proof}
\end{document}
